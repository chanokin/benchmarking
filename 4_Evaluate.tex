\section{Performance Evaluation}
\label{sec:eval}
\subsection{Hardware-Independent}
Most classification papers report a percentage of accuracy that just leaves the reader with a measure of the correct classifications. Should we report also incorrect or ambiguous? Could some ``correct'' be masking ambiguous? Were the ambiguous due to noise? Was the noise added on purpose? Was there any preprocessing done?\\

Another aspect of training using spiking neural networks is the exact exposure of the training data used (how many times each image is presented.)\\

Traditionally, neural network training has been done using rate-based encoding. As new theories emerge, an important distinction to make is the nature of the  training procedure.\\

A brief description of the \emph{network topology} is highly desirable, we feel like different topologies will have a deep impact on the overall performance. Furthermore, sharing this designs can inspire fellow scientists to create new ones. \\

Learning procedures can be classified into two major branches. The main distinction is whether it was done using \emph{supervised} or not. The label of the data helps to establish categories and connection weights in supervised learning. 

Unsupervised learning has less constraints but might be tougher to get right. A number of different classes are expected, this quantity might give an insight onto the network topology and dynamics. What measure did the researchers used to decide classes?\\



\subsection{Hardware-Specific}
	\subsubsection{Training Time}
	\subsubsection{Latency}
	\subsubsection{Power Consumption}
	