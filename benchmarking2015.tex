%%%%%%%%%%%%%%%%%%%%%%%%%%%%%%%%%%%%%%%%%%%%%%%%%%%%%%%%%%%%%%%%%%%%%%%%%%%%%%%%%%%%%%%%%%%%%%%%%%%%%%%%%%%%%%%%%%%%%%%%%%%%%%%%%%%%%%%%
% This is just a template to use when submitting manuscripts to Frontiers, it is not mandatory to use frontiers.cls nor frontiers.tex  %
%%%%%%%%%%%%%%%%%%%%%%%%%%%%%%%%%%%%%%%%%%%%%%%%%%%%%%%%%%%%%%%%%%%%%%%%%%%%%%%%%%%%%%%%%%%%%%%%%%%%%%%%%%%%%%%%%%%%%%%%%%%%%%%%%%%%%%%%


\documentclass[pdftex]{bioinfo}
\usepackage{subfig}
\usepackage{array}
\usepackage{amsmath}
\usepackage{amssymb}
\usepackage{url}
\usepackage{mathptmx}
\usepackage{acronym}
\usepackage{graphicx}
\usepackage{algorithm}
\usepackage{algpseudocode}
\usepackage[colorlinks]{hyperref} % in order to operate correctly, hyperref must be the last package declared
\hypersetup{
  hypertexnames=true, 
  linkcolor=blue,anchorcolor=black,citecolor=blue,urlcolor=blue
}



\newenvironment{equationexp}[1]% Environment for explaining equation variables
{\begin{list}{}%
{\setlength{\leftmargin}{#1}}%
  \item[]%
}
{\end{list}}


\DeclareGraphicsExtensions{.jpg,.pdf,.mps,.png}
\graphicspath{{./images/}} % PUT ALL PDF/JPG/PNG FIGURES IN THIS SUBDIR

% correct bad hyphenation here
\hyphenation{op-tical net-works semi-conduc-tor}

%Commands useful for the review
\newcommand{\revised}[1]{{\color{red} #1}}
\newcommand{\SC}[1]{{\color{red} \textbf{SC: } #1}}


\setlength{\abovecaptionskip}{5pt}
\setlength{\belowcaptionskip}{5pt}

\copyrightyear{2015}
\pubyear{2015}


%%%%%
%\documentclass{frontiersENG} % for Engineering articles
%%\documentclass{frontiersSCNS} % for Science articles
%%\documentclass{frontiersMED} % for Medicine articles
%
%\usepackage{url,lineno}
%\usepackage{epstopdf}
%%\IEEEoverridecommandlockouts
%\usepackage{graphicx}
%\usepackage{subfigure}
%\usepackage{xr-hyper}
%\usepackage{hyperref}
%\linenumbers
%
%% Leave a blank line between paragraphs in stead of using \\
%
%\copyrightyear{}
%\pubyear{}

%%%%%%
%\def\journal{NEUROMORPHIC ENGINEERING}%%% write here for which journal %%%
%\def\DOI{}
%\def\articleType{Research Article}
%\def\keyFont{\fontsize{8}{11}\helveticabold }
\def\firstAuthorLast{Qian Liu {et~al.}} %use et al only if is more than 1 author
%\def\Authors{Qian Liu, Garibaldi Pineda Garcia, Evangelos Stromatias, Teresa Gotarredona, and Steve Furber}
\def\Authors{Qian~Liu\,$^{1,*}$, Garibaldi~Pineda~Garcia\,$^{1}$, Evangelos~Stromatias\,$^{1}$, Teresa~Gotarredona\,$^{2}$, and Steve~Furber\,$^{1}$}
% Affiliations should be keyed to the author's name with superscript numbers and be listed as follows: Laboratory, Institute, Department, Organization, City, State abbreviation (USA, Canada, Australia), and Country (without detailed address information such as city zip codes or street names).
% If one of the authors has a change of address, list the new address below the correspondence details using a superscript symbol and use the same symbol to indicate the author in the author list.
%\def\Address{SpiNNaker, Advanced Processor Technologies Research Group, School of Computer Science, University of Manchester, Manchester, United Kingdom}
\def\Address{$^{1}$SpiNNaker, Advanced Processor Technologies Research Group, School of Computer Science, University of Manchester, Manchester, United Kingdom\\
$^{2}$Instituto de Microelectrónica de Sevilla (IMSE-
CNM-CSIC), Sevilla, Spain }
%% The Corresponding Author should be marked with an asterisk
%% Provide the exact contact address (this time including street name and city zip code) and email of the corresponding author
\def\corrAuthor{Qian Liu}
\def\corrAddress{SpiNNaker, Advanced Processor Technologies Research Group, School of Computer Science, The University of Manchester, Oxford Road, Manchester, M13 9PL, United Kingdom}
\def\corrEmail{qianl.liu-3@manchester.ac.uk}
%
%% \color{FrontiersColor} Is the color used in the Journal name, in the title, and the names of the sections
%

\begin{document}
\firstpage{1}

\title[Benchmarking Spike-Based Visual Recognition: a Dataset and Evaluation]{Benchmarking  Spike-Based Visual Recognition:\\ a Dataset and Evaluation}
\author[\firstAuthorLast ]{\Authors}
\address{\Address}
%\correspondance{}
\history{}

\editor{}


\maketitle
\begin{abstract}
To gain a better understanding of the brain and build biologically-inspired computers, increasing attention is being paid to research into spike-based neural computation.
Within the field, the visual pathway and its hierarchical organisation have been extensively studied within the primate brain.
Spiking Neural Networks (SNNs) inspired by our understanding of observed biological structure and functions have been successfully applied to visual recognition/classification tasks.
A new series of vision benchmarks for spike-based neural processing are now needed to quantitatively measure progress within this rapidly advancing field.
We propose that a large dataset of spike-based visual stimuli is needed to provide a baseline for comparisons.
Furthermore a complementary evaluation methodology is also crucial to assess the accuracy and efficiency of an algorithm.

First of all, an initial dataset of input stimuli based on standard computer vision benchmarks consisting of facial images (FERET database) and digits (MNIST database) is presented according to the current research on spike-based image recognition.
Within this dataset, all images are centre aligned and with similar scale.
We describe how we intend to expand this dataset to fulfil the needs of upcoming research problems.
For instance, the data should provide cases to measure position-, scale-, and viewing-angle invariance.
The data will be in Address-Event Representation (AER) format which is well-applied in neuromorphic engineering field unlike conventional images.
These spike trains are produced by various techniques: rate-based Poisson spike generation, rank order encoding and recorded output from a silicon retina with both flashing and oscillating input stimuli.
An evaluation methodology is also presented which describes how to consistently assess the accuracy, speed, efficiency and cost of an algorithm working with the dataset.
Finally, we provide a baseline for comparison based on a proposed SNN's performance on the dataset.
The network is trained on-line using the Spike Timing Dependent Plasticity (STDP) learning rule on a massive-parallel neuromorphic simulator, e.g. SpiNNaker.

With this benchmark we hope to (1) promote meaningful comparison among algorithms in the field of neural computation, (2) allow comparison with conventional image recognition methods, (3) provide an assessment of the state of the art in spike-based visual recognition, and (4) help researchers identify future directions and advance the field.

\tiny
\section{Keywords:} Benchmarking, Neuromorphic Engineering, Real-Time, Spiking Neural Networks, Vision
%All article types: you may provide up to 8 keywords; at least 5 are mandatory.
\end{abstract}

%\section{Introduction}
%\label{sec:intro}
\section{Introduction}
\label{sec:intro}
%\subsection{What Is the Problem}
With rapid developments in neural engineering, researchers are approaching the aims of understanding brain functions and building brain-like machines using this knowledge~\citep{furber2007neural}.
As a fast growing field, neuromorphic engineering has provided biologically-inspired sensors such as DVS~(Dynamic Vision Sensor) silicon retinas~\citep{serrano2013128, delbruck2008frame, yang2015dynamic, posch2014retinomorphic}, which are good examples of low-cost visual processing thanks to their event-driven and redundancy-reducing style of computation.
Moreover, SNN simulation tools~\citep{davison2008pynn, gewaltig2007nest, goodman2008brian} and neuromorphic hardware platforms~\citep{furber2014spinnaker,  schemmel2010wafer, merolla2014million} have been developed to allow exploration of the brain by mimicking its functions and developing large-scale practical applications~\citep{eliasmith2012large}.
Particularly for visual processing, the central visual system consists of several cortical areas which are placed in a hierarchical pattern according to anatomical experiments~\citep{felleman1991distributed}.
Fast object recognition takes place in  the feed-forward hierarchy of the ventral pathway, one of the two central visual pathways, which mainly handles the ``What'' tasks.
Experiments have revealed that the information is unfolded along the ventral stream to the  IT (Inferior Temporal) cortex~\citep{dicarlo2012does}.
Inspired by the  explicit  biological study of the central visual pathway, SNNs models have successfully been adapted to computer vision tasks. 
%become an active area of computer vision thanks to the
%There are two basic streams locating in the visual area: a dorsal and a ventral pathway.
%They differ in behavioural patterns according to the observation from brain lesions~\citep{prado2005two}, and also in functions where the ventral (`perception') stream perceives the world by means of object recognition and memory, while the dorsal (`action') stream provides real-time visual guidance for motor actions such as eye movements and grasping objects~\citep{goodale1992separate}. 

\cite{riesenhuber1999hierarchical} proposed a quantitative modelling framework of object recognition with position-, scale- and view-invariance based on the units of MAX-like operations.
The cortical-like model has been analysed on several datasets~\citep{serre2007robust}.
And recently~\cite{fu2012spiking} reported that their SNN implementation of the framework was capable of facial expression recognition with a classification accuracy (CA) of 97.35\% on the JAFFE dataset~\citep{lyons1998coding} which contains 213 images of 7 facial expressions posed by 10 individuals.
% 97.35\% on JAFFE dataset.
They employed simple integrate-and-fire neurons with rank order coding (ROC) where  the earliest pre-synaptic spikes have the strongest impact on the post synaptic potentials.
According to~\cite{vanrullen2002surfing}, the first wave of spikes  carry explicit information through the ventral stream and in each stage meaningful information is extracted and spikes are regenerated. 
Using one spike per neuron,~\cite{delorme2001face} reported 100\% and 97.5\% accuracies on the face identification task over changing  contrast and luminance training (40 individuals $\times$ 8 images) and testing data (40 individuals $\times$ 2 images) respectively.
%These developments yielded a large number of papers on SNNs based recognition, with a majority reporting outstanding recognition resulton limited-size databases.

The Convolutional Neural Network (CNN), also known as the \textit{ConvNet} developed by~\cite{lecun1998gradient}, is a well applied model of such a cortex-like framework.
%Reported results:
%Hand Gestures, Qian Liu
An early Convolutional Spiking Neural Network (CSNN) model identified faces of 35 persons with a CA of 98.3\% exploiting simple integrate and fire neurons~\citep{matsugu2002convolutional}.
Another CSNN model~\citep{zhao2014feedforward} was trained and tested both with DVS raw data and Leaky Integrate-and-Fire (LIF) neurons.
%The MAX operation, training and the switch are not only neuron involved.
It was capable of recognising three moving postures with a CA of about 99.48\% and 88.14\% on the MNIST-DVS dataset (see Chapter~\ref{sec:data}).
As one step forward,~\cite{camunas2012event} implemented a convolution processor module in hardware which could be combined with a DVS for high-speed recognition tasks.
The inputs of the ConvNet were continuous spike events instead of static images or frame-based videos. 
The chip detected four suits of a 52 card deck while the cards were fast browsed in only 410 ms.
Similarly, a real-time gesture recognition model~\citep{liu2014real} was implemented on a neuromorphic system with a DVS as a front-end and a SpiNNaker~\citep{furber2014spinnaker} machine as the back-end where LIF neurons built up the ConvNet configured with biological parameters.
In this study's largest configuration, a network of 74,210 neurons and 15,216,512 synapses used 290 SpiNNaker cores in parallel and reached 93.0\% accuracy. 

Deep Neural Networks (DNNs) together with deep learning are the most exciting research fields in vision recognition.
The spiking deep network has great potential to combine remarkable performance with the energy efficient training and running.
In the initial stage of the research, the study was focused on converting off-line trained deep network to SNNs~\citep{o2013real}.
The same network initially implemented on a FPGA achieved a CA of 92.0\%~\citep{neil2014minitaur}, while a later implementation on SpiNNaker scored 95.0\%~\citep{Stromatias2015scalable}.
%The performance was increased from 92.0\% to 95.0\%~\citep{Stromatias2015scalable} by implementing the model on SpiNNaker instead of the earlier FPGA version.
Recent attempts have contributed to better translation by utilising modified units in a ConvNet~\citep{cao2015spiking} and tuning the weights and thresholds~\citep{Diehl2015fast}).
The later paper claims a state-of-the-art performance (99.1\% on the MNIST dataset) comparing to original ConvNet.
The current trend of training Spiking DNNs on-line using biologically-plausible learning methods is also promising.
An event driven Contrastive Divergence (CD) training algorithm for RBMs (Restricted Boltzmann Machines) was proposed for Deep Belief Networks (DBN) using LIF neurons with STDP (Spike-Timing-Dependent Plasticity) synapses and verified on MNIST (91.9\%)~\citep{neftci2013event}.

STDP as a biological learning process is applied to vision tasks.
\cite{bichler2012extraction} demonstrated an unsupervised STDP learning model to classify car trajectories captured with a DVS retina. 
A similar model was tested on a Poissonian spike presentation of the MNIST dataset achieving a performance of 95.0\%~\citep{diehl2015unsupervised}.
Theoretical analysis~\citep{nessler2013bayesian} showed that unsupervised STDP was able to approximate a stochastic version of Expectation Maximization, a powerful learning algorithm in machine learning.
The computer simulation achieved~93.3\% CA on MNIST and could be implemented in a memrisitve device~\citep{bill2014compound}. 

Despite the promising research on SNN-based vision recognition, there is no commonly used database in the format of spikes.
In the studies listed above, all the vision data used are in one of the following formats:
(1) the grey-scale raw values of images;
(2) rate-based spike trains according to pixel intensities created by various Poissonian generators;
(3) unpublished DVS recorded spike-based videos.
As a consequence, a new series of spike-based vision datasets is now needed to quantitatively measure progress within this rapidly advancing field and to provide fair competition resources for researchers.
Apart from using spikes instead of the frame-based data of conventional computer vision, there are new concerns of evaluating neuromorphic vision in tasks other than recognition accuracy.
Therefore a common metric of performance evaluation on spike-based vision is also required to specify the measurements of algorithms and models. 
Different assessments should be taken into consideration when implementing models on neuromorphic hardware, especially the trade-offs between simulation time, precision and power consumption.
Thus benchmarking neuromorphic hardware with various network models will reveal the advantages and disadvantages of different platforms.
In this paper we propose a large dataset of spike-based visual stimuli, NE, and its complementary evaluation methodology.
The dataset expands and evolves as research develops and new problems are introduced.

In Section~\ref{sec:Related}, some example datasets of conventional non-spiking computer vision are introduced.
Section~\ref{sec:guide} defines the purpose and protocols of the proposed dataset.
The sub-datasets and their generation methods are described in detail in Section~\ref{sec:data}.
In accordance with the dataset, its evaluation methodology is demonstrated in Section~\ref{sec:eval}.
Moreover, two SNN models are provided as examples of benchmarking hardware platforms in Section~\ref{sec:test}.
Section~\ref{sec:summ} summarises the paper and discusses future work.
\section{Guiding Principles}
\label{sec:guide}
\subsection{The Goals}
\subsection{Dataset and Testing Protocols Referring the Goals}
\section{The Database: NeuroMorphic15}
\label{sec:data}
\subsection{Current Status}
what are the problems the current database aim for.
Current methods/algorithms existing.
What are the future algorithms we are encouraging.
\subsection{Description}
	Experiment setup/ collection method/ properties of each class/ etc.
	\subsubsection{Poisson}
	\subsubsection{Rank-Order-Encoding}
	\subsubsection{DVS Sensor Output with Flashing Input}
	\subsubsection{DVS Sensor Output with Oscillating Input}
	\subsubsection{DVS Sensor Output with Moving Input}
	
\section{Performance Evaluation}
\label{sec:eval}

When we come to the stage where we have to report results of our research, we often face questions like, \emph{How much should I include in my report? How do I compare my work to others?} This are extremely important questions, and we would like to assist the readers with the following considerations.

\subsection{Hardware-Independent}
A brief description of the \emph{network topology} is most welcome, we believe different topologies will have a deep impact on the overall performance. Furthermore, sharing this designs can inspire fellow scientists to create new structures that can help us see our own from an alternative point of view, generating a positive feedback loop where everybody wins.

Most classification papers report a percentage of accuracy that gives the reader a measure of the correct classifications~[\cite{dietterich1998approximate}]. Some times it might be desirable, for a better understanding of the paper, that a distinction between ambiguous, outliers and incorrect classes is made~[\cite{liu2002performance}]. A very useful piece of information is clear citation of the base-line source, which is almost always there but lost in a sea of references.

%Should we report also incorrect or ambiguous? Could some ``correct'' be masking ambiguous? Were the ambiguous due to noise? Was the noise added on purpose? 

As we are proposing spike based data-sets, it's desirable that the users specify if there was any preprocessing applied before actually feeding the spike trains into their networks~[\cite{best-practice-nn-img}]. For example, if we want to use a particular set to test noisy inputs, it would be extremely useful to have a notion of the type of noise added.

Traditionally, neural network training has been done using rate-based encoding. As new theories emerge, an important distinction to make is the nature of the  training procedure. One of them could be the way training data was exposed to the network (e.g. How many times each image is presented? How much time is a single example shown?. [\cite{unsup_leraning_diehl}]) Details on the learning rules STDP, BCM, 

One the biggest distinctions on learning procedures is whether they were done using some \emph{supervision} or not; making this distinction clear is vastly appreciated. On supervised learning, the label of the data influences to establish categories and connection weights. Unsupervised learning has less constraints when it comes to class creation but might be tougher to get right. 

A number of different classes are expected, this quantity might give an insight onto the network topology and dynamics. A description of the methods used to generate and populate the classes is very helpful for the reader. (e.g. Did we use a statistical measure? Was it a combination of NN with some other algorithms?)


\subsection{Hardware-Specific}
	\subsubsection{Training Time}
	\subsubsection{Latency}
	\subsubsection{Power Consumption}
	

\section{A Test on the Database}
\label{sec:test}
\subsection{Description of the data used}
\subsection{Training}
\subsection{Testing}
\subsection{Evaluation}

	
\section{Conclusion and the Future Work}
\label{sec:summ}
\subsection{What we have said and done}
\subsection{The future direction of developing the database}
What are the future algorithms we are encouraging.

	
\section*{Acknowledgments}
The work presented in this paper is largely inspired on discussions carried out at the 2015 Workshops on Neuromorphic Cognition Engineering in CapoCaccia.
The authors would like to thank the organisers and the sponsors.
The SpiNNaker project is supported by the Engineering and Physical Science Research Council (EPSRC), grant EP/4015740/1, and also by ARM and Silistix. The authors thank the support of these sponsors and industrial partners.
\bibliographystyle{frontiersinSCNS&ENG} % for Science and Engineering articles
%\bibliographystyle{frontiersinMED} % for Medicine articles

\bibliography{ref,rank-ordered,hw-ind-eval}

\end{document}