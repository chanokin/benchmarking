\section{Guiding Principles}
\label{sec:guide}
\subsection{The Goals}
With this benchmark we hope to (1) promote meaningful comparison among algorithms in the field of neural computation, (2) allow comparison with conventional image recognition methods, (3) provide an assessment of the state of the art in spike-based visual recognition, and (4) help researchers identify future directions and advance the field.
\subsection{Dataset and Testing Protocols Referring the Goals}

The FERET database was established to support both
algorithm development and evaluation. Two guiding prin-
ciples were followed. First, the evaluation of algorithms
requires a common database of images for both develop-
ment and testing. In the FERET evaluation, the images in
the test are from both the development
and sequestered
portions of the FERET database. Second, the variety and
difficulty of the problems defined by the images in the data-
base must increase incrementally.
The need to test algorithms against a database is obvious,
but the development
function of the database is equally
important (if less obvious). For the evaluation procedure
to produce meaningful results, the images in the develop-
mental portion of the database must resemble those on
which algorithms are to be tested. The development and
testing data sets must be similar in both quality and quantity.
For example, if the test will consist of a gallery of 1000
individuals,
it is not appropriate
for the development
database to consist of 50 individuals. The algorithms tested
will be only as good as the database from which they are
developed. The FERET evaluation procedure followed this
principle by partitioning the FERET database into the devel-
opmental and sequestered portions, where the developmen-
tal portion was representative
of the sequestered portion
(details are provided in Section 4.2).
The other principle is that the problem defined by the
images in the database must mesh with the current level
of algorithm development, and the difficulty of the database
must grow as the sophistication of the algorithms increases.
As explained in Section 2, if the database defines a problem
that is too easy, testing the algorithm becomes a mere tuning
exercise. At the other extreme, if the problem is too far
beyond the state of the art, the test will not produce any
meaningful results. To prevent the FERET database from
becoming stale, we continuously expanded and adjusted the
database to the state of the art in face recognition.


