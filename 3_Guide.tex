\section{Guiding Principles}
\label{sec:guide}
The NE database we propose here is a developing and evolving dataset consisting of various spike-based representations of images and videos.
The spikes are either generated from spike encoding methods which covert images or frames of videos into spike trains, or recorded from DVS silicon retinas.
The spike trains are in the format of AER data, which could easily be used in both event-driven computer simulations and neuromorphic systems.
With the NE dataset we hope:
\begin{itemize}
	\item \textit{to promote meaningful comparisons of algorithms in the field of spiking neural computation.}
	The NE dataset provides a unified format of AER data to meet the demands of spike-based visual stimuli.
	It also encourages researchers to publish and contribute their data to build up the NE dataset.
	The training and testing sets have to be disjoint and also of similar quality and quantity.
	\item \textit{to allow comparison with conventional image recognition methods.}
	It asks the dataset to support this comparison with spiking versions of existing vision datasets.
	Thus, conversion methods are required to transform datasets of images and frame-based videos to spike stimuli.
	With growing knowledge of biological vision, new methodologies and algorithms are welcomed to present these conventional datasets with spikes in more biological ways. 
%	As the first published subset of the dataset, NE15 is composed of four subsets of spiking versions of MNIST, thus enables recognition performance to be compared with conventional methods. 
	\item \textit{to provide an assessment of the state of the art in spike-based visual recognition on neuromorphic hardware.}
	In order to reveal the advantages of neuromorphic engineering, not only a spike based dataset but also an appropriate evaluation methodology is needed.
	In accordance with the idea of an evolving dataset, the evaluation methodology develops accordingly as a constantly perfected process.
	\item \textit{to help researchers identify future directions and advance the field.}
	The development of the dataset and its evaluation will introduce new challenges to the neuromorphic engineering community.
	However, an easily solved problem turns out to be a tuning competition, while a far beyond difficult problem is not able to bring meaningful assessment.
	So suitable problems should be added continuously to promote the future research.  
\end{itemize}
