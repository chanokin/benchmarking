\section{Related Work}
\label{sec:Related}
In computer vision, there are a few datasets playing import roles in different time and aiming at various targets.
\subsection{MNIST~[\cite{lecun_gradient-based_1998}]}
The MNIST is a subset of NIST hand written digits dataset. 
Both of the training and testing sets are selected half from SD-3 and the other half from SD-1. 
The training set contains 60,000 patterns collected from approximately 250 writers.
And the testing set is composed of 10,000 patterns written by disjoint individuals.
All the digits in the dataset are of similar scale centring in a 28$ \times $28 image.
Due to its straightforward target of classifying real-world images, the plain format of binary data and 
the simple patterns, MNIST is one of the most popular datasets in computer vision over 20 years.

Many methods have been verified on this dataset: K-means, SVM, ConvNets and etc.
The descending recognition error rate makes it nearly a solved problem, however some modifications, such as position shifts, scaling and noise, bring new challenges.
Certainly, a spiking version of the dataset will be an interesting artificial distortion and draw attention to new methods and algorithms on the challenge. 

\subsection{ImageNet~[\cite{deng_imagenet:_2009}]}
Since the new era of the 4th generation ANN, Deep Neural Network, a flow of successful applications have been reported.
Meanwhile, training the deeper network triggers a huge demand for more data.
This is the purpose of putting forward ImageNet to provide researchers a large-scale image database.
Currently there are 14,197,122 images and 21841 synsets indexed in the dataset.
Synsets are meaningful concepts described with a few words or phrases, and they are organised in hierarchy in WordNet.
The final goal of ImageNet is to provide about 1000 images per synset in the total of 100,000 of them.
In another word, there will be tens of millions images tidily structured, accurately labelled and human annotated.
The dataset is a well-recognised benchmark test for the deep learning community, and many attempts have been made to improve the performance, for example~\cite{krizhevsky2012imagenet}.

\subsection{Microsoft COCO~[\cite{lin_microsoft_2014}]}
As a good example of database catching up with state-of-the-art technologies, Microsoft COCO aims to solve three problems in scene understanding by providing a large-scale datasets.
First is to categorise objects in their non-iconic views, such as being small, ambiguous or partially occluded.
Secondly, understanding the context (contextual reasoning) of multiple objects in an image is necessary.
At last, spatial labelling the objects is a core analysis in scene understanding.
Up to date, the dataset contains 300,000+ images, 2 Million instances and 5 captions per image.

\subsection{Action Datasets}
Similar examples could be found in video datasets.
Two early benchmarks, the KTH ~\cite{schuldt2004recognizing} and Weizmann~\cite{blank2005actions}, have been tested on extensively in the past decade. 
These videos were produced with scripted behaviours under controlled environment (``in the lab'') .
And the single atomic actions were simple and neat: walking, running, sitting and etc.

Taking the advantages of continuous spiking trains instead of frames of videos, spiking version of such action datasets will be provided in our future work.
A DVS simulation may be needed to convert frames of images into spikes.
 
YouTube Action Dataset~\cite{liu_recognizing_2009} targets of recognising realistic actions from videos ``in the wild''.
Thanks to the digit era, unconstrained videos are abundant on the Internet, e.g. YouTube.
This YouTube Action dataset is composed of 1,168 videos of 11 categories.
The main challenge relies on the massive variations due to the moving camera, background clutter, viewing angles, illuminations and so on.
It also aims to detect complex action (non-atomic), e.g. long jump, which is consisted with several continuous atomic actions.

%\subsection{Current Status}
%\subsubsection{Converting from MPL to SNN}
%It remains a challenge to transform traditional artificial neural networks into spiking ones.
%There are attempts~\cite{la2008response}~\cite{burkitt2006review} to estimate the output firing rate of the LIF neurons (Equation~\ref{equ:lif}) under certain conditions. 
%%For the model illustrated above, there are two types of synaptic connection: one-to-one connections in the retina layer and N-to-one connections in all the convolutional layers (the pooling layer is also included). 
%%For the retina layer, 1) the problem is: what is the connection weight between two single LIF neurons to make a post-synaptic neuron fire whenever the pre-synaptic neuron generates a spike? 
%%While for the convolutional neurons, 2) given the input spike rates, LIF neuron parameters and the output spiking rate, what are the corresponding weights between the two layers?
%\begin{equation}
%\frac{\D \: V(t)}{\D\:  t}=-\frac{V(t)-V_\mathit{rest}}{\tau_m}+\frac{I(t)}{C_m}
%\label{equ:lif}
%\end{equation}
%The membrane potential $V$ changes in response to input current $I$, starting at the resting membrane potential  $V_{rest}$, where the membrane time constant is $\tau_m = R_mC_m$, $R_m$ is the membrane resistance and $C_m$ is the membrane capacitance.
%
%Given a constant current injection $I$, the response function, i.e. firing rate, of the LIF neuron is
%\begin{equation}
%\lambda_\mathit{out}=
%\left [ t_\mathit{ref}-\tau_m\ln \left ( 1-\frac{V_{th}-V_\mathit{rest}}{IR_m}  \right )\right ]^{-1}
%\label{equ:consI}
%\end{equation}
%when $IR_m>V_{th}-V_{rest}$, otherwise the membrane potential cannot reach the threshold $V_{th}$ and the output firing rate is zero. 
%The absolute refractory period $t_\mathit{ref}$ is included, where all input during this period is invalid.
%In a more realistic scenario, the post-synaptic potentials (PSPs) are triggered by the spikes generated from the neuron's pre-synaptic neurons other than a constant current.
%Assume that the synaptic inputs are Poisson spike trains, the membrane potential of the LIF neuron is considered as a diffusion process. Equation~\ref{equ:lif} can be modelled as a stochastic differential equation referring to Ornstein-Uhlenbeck process,
%\begin{equation}
%\tau_m\frac{\D\:V(t)}{\D\:  t}=-\left[V(t)-V_\mathit{rest}\right] + \mu + \sigma\sqrt{2\tau_m}\xi (t)
%\label{equ:sde}
%\end{equation}
%where
%\begin{equation}
%\begin{array}{l}
%\mu=\tau_m(\mathbf{w_E\cdot\lambda_E}-\mathbf{w_I\cdot\lambda_I})
%\\
%\\
%\sigma ^{2} = \frac{\tau_m}{2}\left(\mathbf{w_E^{2}\cdot\lambda_E}+\mathbf{w_I^{2}\cdot\lambda_I}\right)
%\end{array}
%\label{equ:ou}
%\end{equation}
%are the conditional mean and variance of the membrane potential.
%The delta-correlated process $\xi(t)$ is Gaussian white noise with zero mean, $\mathbf{w_E}$ and $\mathbf{w_I}$ stand for the weight vectors of the excitatory and the inhibitory synapses, and $\mathbf{\lambda}$ represents the vector of the input firing rate.
%The response function of the LIF neuron with Poisson input spike trains is given by the Siegert function~\cite{siegert1951first}, 
%%\begin{equation}
%%%\lambda_\mathit{out}=\left(\tau_\mathit{ref} + \frac{\tau_Q}{\sigma_Q}\sqrt{\frac{\pi}{2}} \int_{V_\mathit{rest}}^{V_\mathit{th}}du \:\exp \left(\frac{u-\mu_Q}{\sqrt2\sigma_Q} \right )^{2}\left[1+erf \left(\frac{u-\mu_Q}{\sqrt2\sigma_Q} \right ) \right ]\right)^{-1}
%%\begin{split}
%%\lambda_\mathit{out}=\left(\tau_\mathit{ref} + \frac{\tau_Q}{\sigma_Q}\sqrt{\frac{\pi}{2}} \int_{V_\mathit{rest}}^{V_\mathit{th}}\D u \:\exp \left(\frac{u-\mu_Q}{\sqrt2\sigma_Q} \right )^{2}\left[1+\mathrm{erf} \left(\frac{u-\mu_Q}{\sqrt2\sigma_Q} \right ) \right ]\right)^{-1}
%%\end{split}
%%\label{equ:sgt}
%%\end{equation}
%\begin{align}
%\lambda_\mathit{out} &=\left(\tau_\mathit{ref} + \frac{\tau_Q}{\sigma_Q}\sqrt{\frac{\pi}{2}} \int_{V_\mathit{rest}}^{V_\mathit{th}}\D\,u \:\exp \left(\frac{u-\mu_Q}{\sqrt2\sigma_Q} \right )^{2} \right. \nonumber \\
%&\qquad \left. \vphantom{\int_t} \cdot  \left[1+\mathrm{erf} \left(\frac{u-\mu_Q}{\sqrt2\sigma_Q} \right ) \right ]\right)^{-1}
%\label{equ:sgt}
%\end{align}
%where $\tau_Q, \mu_Q, \sigma_Q$ are identical to $\tau_m, \mu, \sigma$ in Equation~\ref{equ:ou}, and erf is the error function.
%
%Still there are some limitations on the response function. 
%For the diffusion process, only small amplitude (weight) of the PostSynaptic Potentials (PSPs) generated by a large amount of input spikes (high spiking rate) work under this circumstance; 
%plus, the delta function is required, i.e. the synaptic time constant is considered to be zero. Thus only a rough approximation of the output spike rate has been determined.
%Secondly, given different input spike rate to each pre-synaptic neurons, the parameters of the LIF neuron and the output spiking rate, how to tune every single corresponding synaptic weight remains a difficult task.
%
%\subsubsection{Rank-Order-Coding}
%
%\subsubsection{Liquid State Machine/Reservoir Encoding}
