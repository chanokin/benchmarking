\section{Introduction}
\label{sec:intro}
\subsection{What Is the Problem}
In the last few years, vision based spiking neural networks(SNNs) has become an active area of computer vision.
The interest has been fuelled by potential applications and by the simultaneous development of algorithmic techniques and inexpensive computers with the computational power to run these algorithms.
These developments yielded a large number of papers on SNNs based recognition, with a majority reporting outstanding recognition results (usually $>$ 95\% correct recognition) on limited-size databases (usually $<$ 50 individuals).
%Reported results:
%Hand Gestures, Qian Liu
Inspired by the behaviours of the primary visual cortex, Convolutional Neural Networks (CNNs) are modelled using
both linear perceptrons and spiking Leaky Integrate-and-Fire (LIF) neurons. In this study's largest configuration using these
approaches, a network of 74,210 neurons and 15,216,512 synapses is created and operated in real-time using 290 SpiNNaker
processor cores in parallel and with 93.0% accuracy. A smaller network using only 1/10th of the resources is also created,
again operating in real-time, and it is able to recognize the postures with an accuracy of around 86.4% - only 6.6% lower than
the much larger system. The recognition rate of the smaller network developed on this neuromorphic system is sufficient for a
successful hand posture recognition system, and demonstrates a much improved cost to performance trade-off in its approach.
%Hand Gestures, Lee
The motion trajectories of moving hands are detected by spatiotemporally correlating the stereoscopically verged asynchronous events from the DVSs by using leaky integrate-and-fire (LIF) neurons. Adaptive thresholds of the LIF neurons achieve the segmentation of trajectories, which are then translated into discrete and finite feature vectors. The feature vectors are classified with hidden Markov models, using a separate Gaussian mixture model for spotting irrelevant transition gestures. The disparity information from stereovision is used to adapt LIF neuron parameters to achieve recognition invariant of the distance of the user to the sensor, and also helps to filter out movements in the background of the user. Exploiting the high dynamic range of DVSs, furthermore, allows gesture recognition over a 60-dB range of scene illuminance. The system achieves recognition rates well over 90\% under a variety of variable conditions with static and dynamic backgrounds with naïve users.
%MNSIT Emre Neftci
%The recurrent activity of the network replaces the discrete steps of the CD algorithm, while Spike Time Dependent Plasticity (STDP) carries out the weight updates in an online, asynchronous fashion. We demonstrate our approach by training an RBM composed of leaky I&F neurons with STDP synapses to learn a generative model of the MNIST hand-written digit dataset, and by testing it in recognition, generation and cue integration tasks. Our results contribute to a machine learning-driven approach for synthesizing networks of spiking neurons capable of carrying out practical, high-level functionality.
%MNIST Evangelos Stromatias
In  this  context  we  in-
troduce  a  realization  of  a  spike-based  variation  of  previously
trained  DBNs  on  the  biologically-inspired  parallel  SpiNNaker
platform. The DBN on SpiNNaker runs in real-time and achieves
a classification performance of 95% on the MNIST handwritten
digit  dataset,  which  is  only  0.06%  less  than  that  of  a  pure
software implementation. Importantly, using a neurally-inspired
architecture yields additional benefits: during network run-time
on this task, the platform consumes only 0.3 W with classification
latencies in the order of tens of milliseconds, making it suitable
for  implementing  such  networks  on  a  mobile  platform.  The
results in this paper also show how the power dissipation of the
SpiNNaker platform and the classification latency of a network
scales with the number of neurons and layers in the network and
the overall spike activity rate.


Few of these algorithms reported results on images from a common database; fewer met the desirable goal of being evaluated against a standard testing protocol that includes separate training and testing sets.
As a consequence, there was no way to make a quantitative assessment of the algorithms' relative strengths and weaknesses.
Unfortunately, this is not an isolated case, but an endemic problem in computer vision research.
\subsection{Vision Literature: Datasets Examples}
MNIST
ImageNet
Video sets
\subsection{The Proposal: Advantages}
In this paper we present a comprehensive method of eval-
uating face-recognition
algorithms, developed as part of the
Face Recognition Technology (FERET) program [ 10,111.
The FERET evaluation methodology consists of an inte-
grated data collection effort and testing program. These
two parts are integrated through the FERET database of
facial images; the database is divided into a development
portion, which is provided to researchers, and a sequestered
portion for testing. The partition of the database enables
algorithms to be trained and tested on different, but related,
sets of images. The FERET evaluation procedure is a set of
standard testing protocols: the FERET tests are indepen-
dently administered
and each test is completed within
three days. The use of a standard testing protocol allows
for the direct comparison of algorithms developed by dif-
ferent groups, as well as for measuring improvements made
by any single group over time.